\chapter{Execution}
\label{chap:execution}

{\it GridSim} is executed by using its executable file {\bf gridSim}.
The execution requires a configuration file to configure the experiment that you want to simulate.
The content of this file is explained in Section~\ref{exe:single_cnf}.
It is launch through:
\begin{lstlisting}[language=bash]
  $ ./gridSim -p <CONFIGURATION_FILE>
\end{lstlisting}
\noindent
The simulator is launched during the stipulated time.


%
%************************************************************************************************************
\section{Parallel launcher}

The parallel launcher of GridSim is based on MPI.
The parallel executin must be configured with an independent file.
The content of this file is explained in Section~\ref{exe:parallel_cnf}.
It is launch through:
\begin{lstlisting}[language=bash]
  $ mpirun -np <N_PROCESS> ./gridSim_parallel 
           -p <PARALLEL_sCONFIGURATION_FILE>
\end{lstlisting}
where $<N\_PROCESS>$ is the number of process launched in parallel.
Notice that the parallel execution requires a master process which does not calculate nothing, it is only a job dealer.
Therefore, $<N\_PROCESS>$ should be at least $2$, one master dealing job and a slave simulating.

%
%************************************************************************************************************
\section{Experiment configuration}
\label{exe:single_cnf}

%
The experiments are configured by using the xml configurations files.
These files have the following main structure:
%
\begin{code}
<?xml version="1.0"?>
<GridSim>
	<Simulation 
		seed="SED_NUMBER" 
		length="LENGTH" 
		sampling="SMP_PERIOD" 
		fft_lng="FFT_LENFGTH" 
		data_folder="DATA_FOLDER"
	/>
	<Visualization 
		active="FLAG_ACTIVE" 
		rf_rate="RF_RATE" >
		WINDOWS_DEF			
		...									
	</Visualization>
	<Grid 
		file="GRID_PROFILE" 
		amp="GRID_AMP"
	/>
	<Structure> 
		STRUCTURE_DEF
	</Structure>
	<Writer file="OUTPUT_FILE" />
	<Main_control name="MAIN_CONTROLLER" />
</GridSim>
\end{code} 

%
Table~\ref{tab:exp_conf} shows the definition of the configurable parameters previously defined. 
%
\begin{table}[h]
	\begin{center}	
	\begin{tabular}{|m{3.5cm}|m{10cm}|m{1cm}|} \hline		
		{\bf Name} 	& {\bf Description} 					   & {\bf Data type} 	\\ \hline
		SED\_NUMBER	& The initial random number seed 			   & int		\\ \hline
		LENGTH		& Duration of experiment in real world in minutes	   & int		\\ \hline
		SMP\_PERIOD	& Sampled period for DFT				   & int		\\ \hline
		FFT\_LENFGTH	& DFT window length					   & int		\\ \hline
		DATA\_FOLDER	& Data folder address					   & string		\\ \hline
		FLAG\_ACTIVE	& $0$ deactivate visualization; $1$ activate visualization & bool		\\ \hline
		RF\_RATE	& Refresh rate for visualization			   & int		\\ \hline
		WINDOWS\_DEF    & Definition of windows (see Section~\ref{exe:visu_def})   &			\\ \hline
		GRID\_PROFILE	& Base consumption profile file name			   & string		\\ \hline
		GRID\_AMP	& Amplitude of the base consumption			   & float		\\ \hline
		STRUCTURE\_DEF	& Definition of structure (see Section~\ref{exe:str_def})  & 			\\ \hline
		OUTPUT\_FILE	& Output file name					   & string		\\ \hline
		MAIN\_CTR	& Main controller name					   & string		\\ \hline	
	\end{tabular}
	\caption{Experiment configuration parameters.}
	\label{tab:exp_conf}
	\end{center}
\end{table}


%
%************************************************************************************************************
\subsection{Visualization definition}
\label{exe:visu_def}

%
The windows for visualization are defined with the following structure in the xml file:
%
\begin{code}
	<scr 
		title="TITLE"  
		x_lng="X_LENGTH" 
		y_ini="Y_INI" 
		y_end="Y_END"
	/>	
\end{code}

%
Table~\ref{tab:windows_conf} shows the definition of the configurable parameters previously defined. 
%
\begin{table}[h]
	\begin{center}	
	\begin{tabular}{|m{3.5cm}|m{10cm}|m{1cm}|} \hline		
		{\bf Name} 	& {\bf Description} 			   & {\bf Data type} 	\\ \hline
		TITLE		& Title of the window			   & string		\\ \hline
		X\_LENGTH	& Length of the x-axis in samples	   & int		\\ \hline
		Y\_INI		& Initial y-axis value	                   & int		\\ \hline
		Y\_END		& Final y-axis value    		   & int		\\ \hline	
	\end{tabular}
	\caption{Window configuration parameters.}
	\label{tab:windows_conf}
	\end{center}
\end{table}

%
%************************************************************************************************************
\subsection{Structure definition}
\label{exe:str_def}

%
The structure of the simulated grid is defined with the following structure in the xml file:
%
\begin{code}
	<Structure lines="LINES_NUMBER" > 
		<line_X nodes="NODES_NUMBER" type="NODE_NAME"/>	
		...		
		<NODE_NAME>
			ELEMENTS								
		</NODE_NAME>
	</Structure>	
\end{code}

%
Table~\ref{tab:structure_conf} shows the definition of the configurable parameters previously defined. 
%
\begin{table}[h]
	\begin{center}	
	\begin{tabular}{|m{3.5cm}|m{10cm}|m{1cm}|} \hline		
		{\bf Name} 	& {\bf Description} 			   & {\bf Data type} 	\\ \hline
		LINES\_NUMBER	& Number of lines			   & int		\\ \hline
		line\_X  	& Definition of line X			   & int		\\ \hline
		NODES\_NUMBER	& Number on nodes                   	   & int		\\ \hline
		NODE\_NAME	& Definition of the nodes of the line      & string		\\ \hline
		ELEMENTS	& Elements in the node (explained below)   & 			\\ \hline	
	\end{tabular}
	\caption{Structure configuration parameters.}
	\label{tab:structure_conf}
	\end{center}
\end{table}

The structure of the simulated node is defined with the following structure in the xml file:
%
\begin{code}
	<NODE_NAME>
		<load 
			type_nd = "ND_TYPE" 
			amp_nd  = "ND_AMP" 
			file_nd = "ND_FILE" 
		/>
		<pv   
			type  = "PV_TYPE" 
			gen   = "PV_PROFILE" 
			frc   = "PV_FR_PROFILE" 
			power = "PV_AMP"
		/>
		<bat  
			cap     = "CAP" 
			num_inv = "NUM_INV"
		/>
		<ctr  
			name="CTR_NAME" 
			CTR_CONF			
		/>									
	</NODE_NAME>
\end{code}

%
Table~\ref{tab:node_conf} shows the definition of the configurable parameters previously defined. 
%
\begin{table}[h]
	\begin{center}	
	\begin{tabular}{|m{3.5cm}|m{10cm}|m{1cm}|} \hline		
		{\bf Name} 	& {\bf Description} 			   		        & {\bf Data type} 	\\ \hline
		ND\_TYPE	& No deferrable consumption type (see Table~\ref{tab:ND_conf})	& int		 	\\ \hline
		ND\_AMP		& No deferrable consumption amplitude	   			& float			\\ \hline
		ND\_FILE	& No deferrable consumption file	   			& string		\\ \hline
		PV\_TYPE	& PV type (see Table~\ref{tab:PV_conf})	   			& int			\\ \hline
		PV\_PROFILE	& PV profile file			   			& string		\\ \hline
		PV\_FR\_PROFILE	& PV forecast file			   			& string		\\ \hline
		PV\_AMP		& PV inverter nominal power		   			& float			\\ \hline
		CAP		& Storage system capacity		   			& float			\\ \hline
		NUM\_INV	& Storage system number of inverters	   			& int			\\ \hline
		CTR\_NAME	& Node controller name			   			& string		\\ \hline
		CTR\_CONF	& Configuration of the node controller     			& 			\\ \hline	
	\end{tabular}
	\caption{Node configuration parameters.}
	\label{tab:node_conf}
	\end{center}
\end{table}

%
Table~\ref{tab:ND_conf} shows the different types of no deferrable consumption. 
%
\begin{table}[h]
	\begin{center}	
	\begin{tabular}{|m{3.5cm}|m{12cm}|} \hline		
		{\bf Number} 	& {\bf Description} 			   	        	\\ \hline
		0		& Constant consumption; ND\_AMP indicates its amplitude	  	\\ \hline
		1		& Consumption defined by a file; ND\_FILE indicates the address	\\ \hline
		2		& Consumption defined by the base consumption profile		\\ \hline	
	\end{tabular}
	\caption{No deferrable types.}
	\label{tab:ND_conf}
	\end{center}
\end{table}

%
Table~\ref{tab:PV_conf} shows the different types of PV generation. 
%
\begin{table}[h]
	\begin{center}	
	\begin{tabular}{|m{3.5cm}|m{12cm}|} \hline		
		{\bf Number} 	& {\bf Description} 			   	        	\\ \hline
		0		& PV profile defined by file; PV\_PROFILE indicates the adress	\\ \hline
		1		& PV profile generated by internal model			\\ \hline	
	\end{tabular}
	\caption{No deferrable types.}
	\label{tab:PV_conf}
	\end{center}
\end{table}

%
%************************************************************************************************************
\section{Parallel experiment configuration}
\label{exe:parallel_cnf}


































