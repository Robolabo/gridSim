\chapter{Installation}
\label{chap:installation}


{\it GridSim} is open source simulator of electrical grids released under GPLv3.0.
It can be downloaded from: \href{https://github.com/Robolabo/gridSim.git}{https://github.com/Robolabo/gridSim.git}.
The repository can also be directly downloaded from {\it github} in your computer:
\begin{lstlisting}[language=bash]
  $ git clone https://github.com/Robolabo/gridSim.git
\end{lstlisting}
\noindent
Once the simulator has been downloaded, you can attempt to install it.

{\it GridSim} must be compile in your computer.
It is preapared to be compile and executed in Linux, this manual does not indicate how to install {\it GridSim} in other operating systems.
In addition, this manual describes the instllation process for an Ubuntu distribution.
{\it GridSim} is programmed in {\it C++}, thus, the {\it C++} compiler is required.
The simulator's compilation has been teste for {\it g++} compiler version $4.4$ or higher.
The installation of this compiler is recommended:
\begin{lstlisting}[language=bash]
  $ sudo apt-get install g++
\end{lstlisting}

The compilation of {\it GridSim} is performed by {\it autotools} with {\it libtool}.
Therefore, these tools should be installed in your computer:
\begin{lstlisting}[language=bash]
  $ sudo apt-get install automake autotools-dev libtool
\end{lstlisting}

There is a compilation script which prepare the compilation folder and compile the simulator.
To execute this script:
\begin{lstlisting}[language=bash]
  $ ./build_simulator.sh
\end{lstlisting}
\noindent
It creates the {\bf build} folder.
All compilation files are in this folder.
After the compilation, the executable file is copied in the main folder of {\it GridSim}.
This file is called {\bf gridSim}.

%
%************************************************************************************************************
\section{Parallel launcher compilation}

Different instances of {\it GridSim} can be launched in parallel.
It allows to use the possibilities of a multicore machine or a cluster.
This parallel execution is based on MPI.
Thus, MPI should be installed in your computer:
\begin{lstlisting}[language=bash]
  $ sudo apt-get install openmpi-bin mpi-default-dev
\end{lstlisting}

There is a compilation script for the compilation of the parallel launcher:
\begin{lstlisting}[language=bash]
  $ ./build_simulator_parallel.sh
\end{lstlisting}
\noindent
After the compilation, the executable file is copied in the main folder of {\it GridSim}.
It is called {\bf gridSim\_parallel}.

%
%************************************************************************************************************
\section{Documentation generation}

This manual can be compile from the {\LaTeX} sources.
The {\LaTeX} packages should be installed in your computer:
\begin{lstlisting}[language=bash]
  $ sudo apt-get install texlive-latex-extra texlive-latex-recommended
		 texlive-bibtex-extra texlive-math-extra 
\end{lstlisting}

There is compilation script for the documentation compilation:
\begin{lstlisting}[language=bash]
  $ ./build_doc.sh
\end{lstlisting}
\noindent
After the compilation, the documentation file is copied in the main folder of {\it GridSim}.
It is called {\bf gridsim\_manual.pdf}.


